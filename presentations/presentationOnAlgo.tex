\documentclass[12pt]{beamer}
\usepackage[utf8]{vietnam}
\usepackage{lmodern}
\usepackage{algorithm,algorithmic}
\usepackage{graphicx}
\usetheme{AnnArbor}

\begin{document}
    \author{Trần Lùi Xuống}
    \title{Thuật toán Floyd-Warshall}
    \subtitle{Tìm đường đi ngắn nhất giữa mọi cặp đỉnh}
    %\logo{}
    \institute{fit@hcmus}
    \date{mùa Xuân 2021}
    %\subject{}
    %\setbeamercovered{transparent}
    %\setbeamertemplate{navigation symbols}{}
    \begin{frame}[plain]
        \maketitle
    \end{frame}

    \begin{frame}
    \frametitle{Nội dung chính}
    \tableofcontents
    \end{frame}

    \begin{frame}
        \frametitle{Bài toán đường đi ngắn nhất}
        \begin{itemize}
            \item Trong một đồ thị không trọng số, đường đi ngắn nhất giữa 2 đỉnh = min(số lượng cạnh phải đi qua).
            \item Trong đồ thị có trọng số, đường đi ngắn nhất là đường đi có tổng trọng số đạt min.
            \item Thuật toán Dijkstra là một thuật toán hiệu quả để giải quyết bài toán đường đi ngắn nhất trên đồ thị có trọng số.
        \end{itemize}
    \end{frame}

    \begin{frame}
        \frametitle{Cung âm}
        \begin{itemize}
            \item Một cạnh trong đồ thị (có hướng hoặc vô hướng) mang trọng số âm được gọi là \textit{cung âm}.
            \item Một nhược điểm của thuật toán Dijkstra: cho kết quả sai khi đồ thị có cung âm!
            \item Để khắc phục, ta phải sử dụng một thuật toán khác: Bellman-Ford hoặc Floyd-Warshall.
        \end{itemize}
    \end{frame}

    \begin{frame}[t]
        \frametitle{Giới thiệu thuật toán Floyd-Warshall}
        \section{Giới thiệu}

        \begin{itemize}
            \item Được phát biểu riêng lẻ bởi Stephen Warshall năm 1959 và Robert W. Floyd năm 1962.
            \item Là một thuật toán Quy hoạch động tiêu biểu.
        \end{itemize}

        \begin{minipage}{0.45\textwidth}
        \begin{figure}
            \includegraphics[scale=0.5]{algo/floyd.png}
            \caption{Robert W. Floyd}
        \end{figure}
        \end{minipage}
        \begin{minipage}{0.45\textwidth}
        \begin{figure}
            \includegraphics[scale=2]{algo/warshall.jpg}
            \caption{Stephen Warshall}
        \end{figure}
        \end{minipage}
    \end{frame}

    \section{Ý tưởng}
    \begin{frame}
        \frametitle{Ý tưởng}
        \framesubtitle{Công thức truy hồi}
        Xét một đồ thị vô hướng hoặc có hướng.

        Gọi F[A][B] là độ dài đường đi ngắn nhất từ đỉnh A đến đỉnh B.\pause
        \begin{itemize}
            \item $F[A][A] = 0$, từ một đỉnh không tốn chi phí để đi đến chính nó.\pause
            \item $F[A][B] = k$, nếu có một cung nối trực tiếp A và B mang trọng số $k$.\pause
            \item $F[A][B] = \infty$, nếu không có cung nối A và B.\pause
            \item $F[A][B] = min(F[A][K] + F[K][B], F[A][B])$, với K là một đỉnh trung gian có cung nối từ A đến K và từ K đến B.
        \end{itemize}
    \end{frame}

    \begin{frame}
        \frametitle{Ý tưởng}
        \framesubtitle{Minh họa}

        (đồ thị ở đây lul)
    \end{frame}

    \begin{frame}
        \frametitle{Ý tưởng (cont.)}
        \framesubtitle{Nhận xét}
        \textbf{Nhận xét}: Gọi F là mảng sau khi đã tính toán bằng thuật toán Floyd-Warshall.\pause
        \begin{itemize}
            \item Thuật toán Floyd-Warshall tính toán đường đi giữa tất cả các cặp đỉnh. Khi cần sử dụng, ta chỉ cần gọi F[<đỉnh đi>][<đỉnh đến>].\pause
            \item Thuật toán Floyd-Warshall còn có thể sử dụng để nhận biết chu trình âm: $F[A][B] < 0 \iff $ giữa A và B tồn tại một chu trình âm.\footnote{chu trình âm là chu trình có tổng trọng số âm}\cite{gtlt:book}
        \end{itemize}
    \end{frame}

    \section{Cài đặt}
    \begin{frame}[t]
        \frametitle{Cài đặt}
        \framesubtitle{Khởi tạo mảng $F$ lưu đường đi ngắn nhất giữa mọi cặp đỉnh}
        \begin{algorithm}[H]
            \begin{algorithmic}[1]
                \FOR{$i=1$ to $N$}
                    \STATE $F[i][i] = 0$
                \ENDFOR
            \end{algorithmic}
            \caption{Khởi tạo đường đi ngắn nhất cho từng đỉnh tới chính nó}
        \end{algorithm}
    \end{frame}
    \begin{frame}[t]
        \frametitle{Cài đặt (cont.)}
        \framesubtitle{Khởi tạo mảng $F$ lưu đường đi ngắn nhất giữa mọi cặp đỉnh}
        \begin{algorithm}[H]
            \begin{algorithmic}[1]
                \FOR{$i=1$ to $N$}
                \FOR{$j=1$ to $N$}
                \IF {$i \neq j$}
                \IF {có cung nối $i$ và $j$ mang trọng số $k$}
                \STATE $F[i][j] = k$
                \ELSE
                \STATE $F[i][j] = \infty$
                \ENDIF
                \ENDIF
                \ENDFOR
                \ENDFOR
            \end{algorithmic}
            \caption{Khởi tạo đường đi ban đầu giữa các đỉnh}
        \end{algorithm}
    \end{frame}
    \begin{frame}[t]
        \frametitle{Cài đặt (cont.)}
        \framesubtitle{Tính toán đường đi ngắn nhất giữa các cặp đỉnh}
        \begin{algorithm}[H]
            \begin{algorithmic}[1]
                \FOR{$k=1$ to $N$}
                \FOR{$i=1$ to $N$}
                \FOR{$j=1$ to $N$}
                \STATE $F[i][j] = min(F[i][k] + F[k][j], F[i][j])$
                \ENDFOR
                \ENDFOR
                \ENDFOR
            \end{algorithmic}
            \caption{Tính toán đường đi ngắn nhất giữa các cặp đỉnh}
            \label{alg:seq}
        \end{algorithm}
    \end{frame}
    \begin{frame}[t]
        \frametitle{Cài đặt (cont.)}
        \framesubtitle{Truy vết}
        Do thuật toán Floyd-Warshall là một thuật toán quy hoạch động nên ta có thể áp dụng phương pháp truy vết:
    \end{frame}
    \begin{frame}
        \frametitle{Cài đặt (cont.)}
        \framesubtitle{Demo code C++}
    \end{frame}


    \section{Đánh giá}
    \begin{frame}[t]

        \frametitle{Đánh giá}
        \framesubtitle{Độ phức tạp thời gian}
        \begin{itemize}
            \item Độ phức tạp trong trường hợp tốt nhất: $\mathcal{O}(n^3)$
            \item Độ phức tạp trong trường hợp tệ nhất: $\mathcal{O}(n^3)$
            \item Độ phức tạp trung bình: $\mathcal{O}(n^3)$
        \end{itemize}
        \textbf{Nhận xét:} Thuật toán Floyd-Warshall có độ phức tạp khá \textbf{tệ} nhưng độ phức tạp mỗi truy vấn đường đi ngắn nhất sẽ là $\mathcal{O}(1)$.
    \end{frame}

    \begin{frame}[t]
        \frametitle{Đánh giá (cont.)}
        \framesubtitle{Độ phức tạp không gian}
        \begin{itemize}
            \item Độ phức tạp trong trường hợp tốt nhất: $\mathcal{O}(n^2)$
            \item Độ phức tạp trong trường hợp tệ nhất: $\mathcal{O}(n^2)$
            \item Độ phức tạp trung bình: $\mathcal{O}(n^2)$
        \end{itemize}
    \end{frame}

    \begin{frame}
        \frametitle{Tổng kết}
        \begin{itemize}
            \item Với độ phức tạp trung bình và tệ nhất đều là $\mathcal{O}(n^3)$, thuật toán Floyd nên được cân nhắc khi sử dụng trong các bài toán đồ thị.
            \item Có một cải tiến của thuật toán Floyd khiến độ phức tạp trung bình giảm từ $\mathcal{O}(n^3)$ xuống còn $\mathcal{O}(n^{3 - \epsilon})$
        \end{itemize}
    \end{frame}

    \begin{frame}[t]
        \frametitle{Cải tiến}
        Một cải tiến được H.Grag và P.Rawat đưa ra năm 2012\cite{improve:article}, theo đó làm giảm độ phức tạp xuống một cách \textbf{đáng kể}.
    \end{frame}

    \begin{frame}
        \Large \centering
        The End!
    \end{frame}

    \section{Tài liệu tham khảo}
    \begin{frame}[t]
        \frametitle{Tài liệu tham khảo}

        \bibliographystyle{apalike}
        \bibliography{bib/algo}
    \end{frame}
\end{document}