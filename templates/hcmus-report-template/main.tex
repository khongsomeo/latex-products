\documentclass[12pt]{article}

\usepackage{amsmath}
\usepackage{amsfonts}
\usepackage{float}
\usepackage{fancyhdr}
\usepackage{graphicx}
\usepackage[colorlinks=true,linkcolor=blue, citecolor=red]{hyperref}
\usepackage{url}
\usepackage[top=.75in, left=.75in, right=.75in, bottom=1in]{geometry}
\usepackage[utf8]{vietnam}

% For algorithm
\usepackage{algorithm}
\usepackage{algpseudocode}

% For code
\usepackage{listings}
\usepackage{xcolor}
\definecolor{codegreen}{rgb}{0,0.6,0}
\definecolor{codegray}{rgb}{0.5,0.5,0.5}
\definecolor{codepurple}{rgb}{0.58,0,0.82}
\definecolor{backcolour}{rgb}{0.95,0.95,0.92}

\lstdefinestyle{mystyle}{
    backgroundcolor=\color{backcolour},   
    commentstyle=\color{codegreen},
    keywordstyle=\color{magenta},
    numberstyle=\tiny\color{codegray},
    stringstyle=\color{codepurple},
    basicstyle=\ttfamily\footnotesize,
    breakatwhitespace=false,         
    breaklines=true,                 
    captionpos=b,                    
    keepspaces=true,                 
    numbers=left,                    
    numbersep=5pt,                  
    showspaces=false,                
    showstringspaces=false,
    showtabs=false,                  
    tabsize=2
}
\lstset{style=mystyle}

% Header length

\setlength{\headheight}{29.43912pt}

% \graphicspath{PATH_TO_GRAPHIC_FOLDER}

% To use Times font family, uncomment this row
% \usepackage{mathptmx}

% Footer page number would be on the lower-right corner
\pagestyle{fancy}
\fancyfoot{}
\fancyfoot[R]{Page \thepage}

\lhead{
Tên báo cáo
}
\rhead{
Trường Đại học Khoa học Tự nhiên - ĐHQG HCM\\
Tên môn học
}
\lfoot{\LaTeX\ by \href{https://github.com/trhgquan}{Quan, Tran Hoang}}

\newcommand{\coursename}{Môn Học Gì Đấy}
\newcommand{\reportname}{Tên Báo Cáo Gì Đấy}

\begin{document}

\begin{titlepage}
\newcommand{\HRule}{\rule{\linewidth}{0.5mm}}
\centering

\textsc{\LARGE đại học quốc gia tphcm}\\[1.5cm]
\textsc{\Large trường đại học khoa học tự nhiên}\\[0.5cm]
\textsc{\large khoa công nghệ thông tin}\\[0.5cm]
\textsc{bộ môn công nghệ tri thức}\\[0.5cm]

\HRule \\[0.4cm]
{ 
\huge{\bfseries{Báo cáo Bài tập gì gì đấy}}\\[0.5cm]
\large{\bfseries{Đề tài: \reportname}}
}\\[0.4cm]
\HRule \\[0.5cm]

\textbf{\large Môn học: \coursename}\\[0.5cm]

\begin{minipage}[t]{0.4\textwidth}
\begin{flushleft} \large
\emph{Sinh viên thực hiện:}\\
Trần Hoàng Tử (19120338)
% Nguyễn Văn A \textsc{(19120000)}
\end{flushleft}
\end{minipage}
~
\begin{minipage}[t]{0.4\textwidth}
\begin{flushright} \large
\emph{Giáo viên hướng dẫn:} \\
% Dr. James \textsc{Smith}
GS. TS. Nguyễn Văn Hướng Dẫn
\end{flushright}
\end{minipage}\\[2cm]

{\large \today}\\[2cm]

\includegraphics[scale=.3]{img/hcmus-logo.png}\\[1cm] 

\vfill
\end{titlepage}
	
	
\tableofcontents
\pagebreak

\section{Section}
Ngày xửa ngày xưa, ở vương quốc VNUHCM - US, có một chàng hoàng tử ngồi cắm đầu viết doc\cite{greenwade93}\footnote{Đây là footnote, chú thích lại những gì cần chú ý. \textbf{Citation để ví dụ cho vui}}.\\
Mặc định muốn xuống dòng chỉ cần dùng $\backslash\backslash$  (2 lần dấu xẹt huyền). 

Nếu muốn xuống dòng nhanh bằng enter 2 lần mà không thụt đầu dòng, thì thêm \textit{$\backslash$noindent} vào trước.

\noindent Đây là ví dụ, xuống dòng và dùng noindent.

\subsection{First subsection}
\subsubsection{First sub-subsection}
Subsection để ví dụ thôi. Thêm vài ví dụ:
\begin{itemize}
    \item Dùng itemize
    \item Vẫn là itemize
\end{itemize}
Sau đó xài enumerate:
\begin{enumerate}
    \item Dùng enumerate
    \item Vẫn là enumerate
\end{enumerate}

\section{Hình ảnh}
Hình ảnh được thể hiện như hình~\ref{fig:my_label}, lưu ý flag \texttt{[H]} để disable floating (hình được hiển thị đúng vị trí, không trôi lên đầu trang).
\begin{figure}%[H]
    \centering
    \includegraphics[scale=.4]{img/hcmus-logo.png}
    \caption{Hình ví dụ (logo HCMUS - updated 30/11/2022)}
    \label{fig:my_label}
\end{figure}

\section{Bảng biểu}
Bảng biểu được thể hiện như bảng~\ref{tab:my_label}, lưu ý flag \texttt{[H]} để disable floating (bảng được hiển thị đúng vị trí, không trôi lên đầu trang).
\begin{table}%[H]
\centering
\begin{tabular}{|l|l|}
\hline
\textbf{Tên con vật} & \textbf{Số chân} \\ \hline
Gà & 2 \\ \hline
Chó & 4 \\ \hline
Trần Hoàng Tử & 2 \\ \hline
\end{tabular}
\caption{Số chân của một số con vật}
\label{tab:my_label}
\end{table}
Để không phải mất thời gian tuổi trẻ ngồi chỉnh table, xài \href{https://www.tablesgenerator.com}{https://www.tablesgenerator.com}.

\section{Công thức toán}
Công thức toán gõ chung 1 dòng thì dùng 2 lần dấu dollar: $f(x) = x^2 + 2x + 1$. Với công thức nằm riêng 1 dòng thì gõ 2 cặp dấu dollar:
$$
ReLU(x) = \max(0, x)
$$
Siêu việt hơn, gõ hệ phương trình thì nên dùng tag \texttt{equation}
\begin{equation*}
\begin{aligned}
a_1x_1 + a_2x_2 + .. + a_nx_n &= u \\
b_1x_1 + b_2x_2 + .. + b_nx_n &= v \\
c_1x_1 + c_2x_2 + .. + c_nx_n &= w \\
\end{aligned}
\end{equation*}
Tham khảo cách gõ equation ở Overleaf nhé!

\section{Thuật toán}
Dùng gói \texttt{algorithm} và \texttt{algpseudocode} để gõ đoạn thuật toán~\ref{alg:label}\footnote{Tất nhiên đây là dùng katana mổ ruồi!}

\begin{algorithm}
\caption{Thuật toán đếm xem nhiều gà hay nhiều chó hơn}
\label{alg:label}
\begin{algorithmic}
\Function {GaChoSoNaoLonHon}{\textit{ga}, \textit{cho}}
\State $soGa \gets 0$
\State $soCho \gets 0$
\For {$i \in [0, |soGa| - 1]$}
\State $soGa \gets soGa + 1$
\EndFor
\For {$i \in [0, |soCho| - 1]$}
\State $soCho \gets soCho + 1$
\EndFor

\If {$soGa > soCho$}
\State\Return $soGa$
\EndIf
\State\Return $soCho$
\EndFunction
\end{algorithmic}
\end{algorithm}

\section{Code}
Dùng gói \texttt{listings} để gõ code, ví dụ cho Python:

\begin{lstlisting}[language=Python]
print("Hello world!")
\end{lstlisting}

\noindent Cho C++:
\begin{lstlisting}[language=C++]
#include<iostream>

int main() {
    std::cout << "Hello, world!\n";
    return 0;
}
\end{lstlisting}
\cleardoublepage
\phantomsection
\addcontentsline{toc}{section}{Tài liệu}
\bibliographystyle{plain}
\bibliography{sample}

\end{document}