\section{Section}

\subsection{Một số lưu ý}
Template này yêu cầu cài đặt một số gói (package) nâng cao:
\begin{itemize}
\item Để gõ thuật toán: \texttt{algorithm} và \texttt{algpseudocode}
\item Để nhúng (chèn) code: \texttt{listings}
\end{itemize}
Các gói này được cài đặt thông qua lệnh
\begin{lstlisting}[language=sh]
sudo apt-get install texlive-full
\end{lstlisting}
Tuy nhiên kích thước gói đâu đó vào khoảng 5GB (!). Vì vậy tốt nhất nên xài Overleaf.

\subsection{Ví dụ}
Ngày xửa ngày xưa, ở vương quốc VNUHCM - US, có một chàng hoàng tử ngồi cắm đầu viết doc\cite{greenwade93}\footnote{Đây là footnote, chú thích lại những gì cần chú ý. \textbf{Citation để ví dụ cho vui}}.\\
Mặc định muốn xuống dòng chỉ cần dùng $\backslash\backslash$  (2 lần dấu xẹt huyền).\\
Nếu bạn muốn thụt đầu dòng khi bắt đầu paragraph mới, vào \texttt{main.tex} và disble dòng
\begin{lstlisting}[language=tex]
\setlength{\parindent}{0pt}
\end{lstlisting}


\subsection{First subsection}
\subsubsection{First sub-subsection}
Subsection để ví dụ thôi. Thêm vài ví dụ:
\begin{itemize}
    \item Dùng itemize
    \item Vẫn là itemize
\end{itemize}
Sau đó xài enumerate:
\begin{enumerate}
    \item Dùng enumerate
    \item Vẫn là enumerate
\end{enumerate}
Nhỏ hơn subsubsection thì xài \texttt{paragraph}:

\paragraph{Đây là ví dụ cho paragraph}
Lưu ý là paragraph không nằm trong Mục lục.