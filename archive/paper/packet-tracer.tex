\documentclass[]{article}

\usepackage{fancyhdr}
\usepackage{float}
\usepackage{graphicx}
\usepackage[colorlinks=true,linkcolor=blue]{hyperref}
\usepackage{lastpage}
\usepackage{listings}
\usepackage[margin=1in]{geometry}
\usepackage[nottoc,notlot,notlof]{tocbibind}
\usepackage[utf8]{vietnam}
\usepackage{url}
\usepackage{xcolor}
\usepackage{booktabs}

\lstset{
    frame = single,
    basicstyle = \ttfamily,
    breaklines = true,
    basicstyle=\footnotesize\ttfamily,
}

\graphicspath{{./image/packetTracer/}}

\pagestyle{fancy}
\lhead{Báo cáo Đồ án 3 - Packet Tracer}
\rhead{Trường Đại học Khoa học Tự nhiên - ĐHQG HCM}
\lfoot{\LaTeX\ by \href{https://github.com/trhgquan}{Quan, Tran Hoang}.}

% Title Page
\title{\textbf{Báo cáo Đồ án 3 - Packet Tracer}}
\author{Nhóm sinh viên lớp 19CTT2}
\date{mùa Xuân năm 2021}

\renewcommand\arraystretch{1.5}

\begin{document}
\maketitle
\tableofcontents
%\listoffigures
\listoftables
\pagebreak

\section{Thông tin}
\subsection{Nhóm sinh viên thực hiện}
\begin{table}[ht]
    \centering
    \caption{Các sinh viên thực hiện Đồ án 3 - Packet Tracer.}
    \begin{tabular}[t]{lcl}
        \toprule
        Họ \& tên&MSSV&Vai trò\\
        \midrule
        Trần Hoàng Quân&19120338&Cài đặt Lab 1;\\
        &&Cài đặt Lab 3;\\
        &&Viết báo cáo;\\
        Lâm Hải Triều&19120407&Cài đặt Lab 2;\\
        &&Quay video demo;\\
        &&Screenshot cho báo cáo;\\
        \bottomrule
    \end{tabular}
\end{table}

\subsection{Phần mềm sử dụng}
Nhóm sử dụng \href{https://www.netacad.com/courses/packet-tracer}{phần mềm Cisco Packet Tracer} \texttt{phiên bản 7.3.1}. Giao diện phần mềm có thể khác với các phiên bản cũ hơn.

\section{Lab 1}
\subsection{Cách cài đặt}
Để cài đặt Lab 1, nhóm đã thực hiện theo các bước sau:
\begin{enumerate}
\item Chọn Router, Switch và PC, sau đó nối dây theo đề bài.
\begin{figure}[H]
    \centering
    \includegraphics[scale=0.5]{lab1/Lab01-hinh1.png}
    \caption{Thực hiện cài đặt các thành phần của mạng và đi dây}
\end{figure}
\item Thực hiện cấu hình Router bằng các lệnh đã cung cấp.
\begin{itemize}
\item Chọn \texttt{Router > Config}
\item Load nội dung config như đề bài đã cho:
\begin{lstlisting}
Router(config)# int fa0/0
Router (config-if)#ip address 192.168.10.1 255.255.255.0
Router (config-if)#exit
Router (config-if)#ip dhcp pool dhclab
Router(dhcp-config)# network 192.168.10.0 255.255.255.0
//network network-number [mask | /prefix-length]
Router(dhcp-config)#default-router 192.168.10.1
Router(dhcp-config)#dns-server 192.168.10.2
Router(dhcp-config)# exit
Router(config)# ip dhcp excluded-address 192.168.10.1 192.168.10.10
//not in the dhcp pool
Router(config)#ip dhcp excluded-address 192.168.10.248 192.168.10.254
//not in the dhcp pool
Router(config)#exit
Router(config)#wr //set the all PC network configuration to DHCP
\end{lstlisting}
\begin{figure}[H]
    \centering
    \includegraphics[scale=0.8]{lab1/Lab01-hinh2.png}
    \caption{Load config của Router.}
\end{figure}
\end{itemize}
\item \texttt{Truy cập từng PC > Desktop > IP Configuration}. Tick phần \texttt{DHCP} để chỉnh IP sang IP cấp phát động.\label{checkDHCP}
\begin{figure}[H]
    \centering
    \includegraphics[scale=0.5]{lab1/Lab01-hinh3.png}
    \caption{Chỉnh IP mỗi PC từ Static sang DHCP}
\end{figure}
\end{enumerate}

\subsection{Kiểm thử}
\subsubsection{Kiểm tra xem IP các PC có được cấp phát DHCP hay không}\label{capPhatDHCP}
Từ thao tác \ref{checkDHCP}, nếu phần IP có câu \texttt{DHCP request successful} thì ta có thể xác nhận IP của PC được cấp phát DHCP.
\begin{figure}[H]
    \centering
    \includegraphics[scale=0.5]{lab1/Lab01-hinh4.png}
    \caption{Xác nhận trạng thái cấp phát IP của mỗi PC}
\end{figure}

\subsection{Báo cáo}
Đề bài nêu 3 câu hỏi:
\begin{enumerate}
    \item \textbf{Địa chỉ IP sau khi được cấp phát DHCP} của PC0, PC1 và PC2?
    \item \textbf{Địa chỉ Gateway} của PC0, PC1 và PC2?
    \item Server nào là \textbf{DNS Server} của PC0, PC1 và PC2?
\end{enumerate}
Sau khi cài đặt hệ thống mạng, nhóm đã thu được các kết quả sau đây:
\begin{figure}[H]
    \centering
    \includegraphics[scale=0.5]{lab1/Lab01-hinh7.png}
    \caption{Thông tin IP PC0 - Lab1}
\end{figure}
\begin{figure}[H]
    \centering
    \includegraphics[scale=0.5]{lab1/Lab01-hinh5.png}
    \caption{Thông tin IP PC1 - Lab1}
\end{figure}
\begin{figure}[H]
    \centering
    \includegraphics[scale=0.5]{/lab1/Lab01-hinh6.png}
    \caption{Thông tin IP PC2 - Lab1}
\end{figure}
Các kết quả có thể được tổng hợp lại trong bảng sau:
\begin{table}[H]
    \centering
    \caption{Kết quả các thử nghiệm ở Lab1.}
    \begin{tabular}[t]{lccc}
        \toprule
        PC&IPv4 Address&Default Gateway&DNS Server\\
        \midrule
        PC0&192.168.10.11&192.168.10.1&192.168.10.2\\
        PC1&192.168.10.13&192.168.10.1&192.168.10.2\\
        PC2&192.168.10.12&192.168.10.1&192.168.10.2\\
        \bottomrule
    \end{tabular}
\end{table}

\section{Lab 2}
\subsection{Cách cài đặt}
Để cài đặt Lab2, nhóm đã thực hiện theo các bước sau:
\begin{enumerate}
\item Chọn Router, Switch và PC theo mô hình đề bài, sau đó tiến hành nối dây.
\begin{figure}[H]
    \centering
    \includegraphics[scale=0.4]{lab2/Lab02-hinh1.png}
    \caption{Mô hình mạng trong Lab2}
    \label{fig:MohinhLab2}
\end{figure}
\item Cấu hình IP Static cho các PC.\label{cauhinhippc}
\begin{itemize}
    \item Ở mỗi \texttt{PC > Desktop > IP Configuration}.
    \item Nhập \texttt{IP Address, Subnet Mask} và \texttt{Default Gateway} theo đề bài.
    \begin{figure}[H]
        \centering
        \includegraphics[scale=0.5]{lab2/Lab02-hinh2.png}
        \caption{Cấu hình static IP cho PC}
    \end{figure}
\end{itemize}
\item Cấu hình Router mặc định cho mạng.
Ở đây ta tiến hành cấu hình cho Router0, tương tự với Router1.\label{chRouter}
\begin{itemize}
\item Vào chế độ \texttt{CLI}\footnote{IOS Command Line Interface} của Router và nhập:
\begin{lstlisting}
Router0>enable
Router0#conf t
\end{lstlisting}
\item Do ta đang xét cổng \texttt{Fa0/0} là cổng nhận các gói tin từ \textbf{trong} mạng, nên ta cần mở cổng \texttt{Fa0/0} và đặt IP của Router trong mạng là \texttt{192.168.1.1/24}:
\begin{lstlisting}
Router0#(config) interface fa0/0
Router0#(config-if) ip-address 192.168.1.1 255.255.255.0
\end{lstlisting}
\item Đặt IP mặc định cho mạng. Ở đây, cổng \texttt{fa0/0} nối với mạng \texttt{192.168.1.0/24}:
\begin{lstlisting}
Router0#(config-if) network 192.168.1.0 255.255.255.0
\end{lstlisting}
\item Sau đó, đặt Router mặc định cho mạng \texttt{192.168.1.0/24}:
\begin{lstlisting}
Router0#(config-if) default-router 192.168.1.1
\end{lstlisting}
\item Cuối cùng, đặt trạng thái cho cổng \texttt{fa0/0} luôn mở và ghi lại các thay đổi của Router.
\begin{lstlisting}
Router0#(config-if) no shutdown
Router0#(config-if) end
Router0# wr
\end{lstlisting}
\end{itemize}
Kết thúc bước này, các máy tính trong mạng \texttt{192.168.1.0/24} đã có thể \texttt{ping} cho nhau.
\begin{figure}[H]
    \centering
    \includegraphics[scale=0.6]{lab2/Lab02-hinh3.png}
    \caption{PC0 đã ping thành công cho PC1}
\end{figure}
\item Cấu hình 2 Router chuyển hướng gói tin đến nhau.\label{chRouting}
\begin{itemize}
\item Xem Hình \ref{fig:MohinhLab2}, ta nhận thấy Router0 và Router1 nằm chung một đường mạng \texttt{192.168.3.4/30}, vì vậy ta tiến hành cấu hình cho cả Router0 và Router1 như ở bước \ref{chRouter}.
\item Sau khi cấu hình, mở \texttt{CLI}. Ta sẽ tiến hành định hướng gói tin cho Router0, Router1 cấu hình tương tự.
\item Router0 đang \textbf{nhận các gói tin} từ mạng \texttt{192.168.1.0/24\footnote{Mask: 255.255.255.0}}; Router1 \textbf{nhận các gói tin gửi đến} ở cổng \texttt{fa0/1} có IP là \texttt{192.168.3.6}.
\begin{lstlisting}
Router0#conf t
Router0#(config) ip route 192.168.1.0 255.255.255.0 192.168.3.6
Router0#(config) end
Router0# wr
\end{lstlisting}
\item Tương tự với Router1:
\begin{lstlisting}
Router1#conf t
Router1#(config) ip route 192.168.2.0 255.255.255.0 192.168.3.5
Router1#(config) end
Router1# wr
\end{lstlisting}
\end{itemize}
\end{enumerate}
\subsection{Kiểm thử}
\subsubsection{Kiểm thử các PC có thể tương tác với nhau hay không}
\begin{enumerate}
\item Chọn 2 PC bất kỳ cùng mạng và \texttt{ping} đến PC còn lại.

Cách thực hiện thao tác \texttt{ping}, xem ở phần \ref{Lab02-task1}
\item Chọn 2 PC bất kỳ khác mạng và \texttt{ping} đến PC còn lại.

(Xem kết quả ở phần \ref{Lab02-task1})
\end{enumerate}

\subsection{Báo cáo}
\subsubsection{Thực hiện và chụp screenshot thao tác \texttt{ping} giữa PC0 và PC2, PC3 và PC1}\label{Lab02-task1}
\begin{enumerate}
\item Để \texttt{ping} giữa 2 máy, ở giao diện PC ta chọn \texttt{Desktop > Command Prompt}\label{howToPing}
\begin{figure}[H]
    \centering
    \includegraphics[scale=0.5]{lab2/Lab02-hinh4.png}
    \caption{Chọn giao diện Command Prompt}
\end{figure}
\item Sau khi giao diện \texttt{Command Prompt} hiện ra, dùng lệnh \texttt{ping <IP máy đích>} để tiến hành \texttt{ping} đến địa chỉ IP máy đích
\end{enumerate}
Dưới đây là kết quả thao tác \texttt{ping} giữa PC0 và PC2, PC3 và PC1.
\begin{figure}[H]
    \centering
    \includegraphics[scale=0.5]{lab2/Lab02-Hinh5.png}
    \caption{Kết quả \texttt{ping} từ PC0 \texttt{(192.168.1.2)} đến PC2 \texttt{(192.168.2.2)}}
\end{figure}
\begin{figure}[H]
    \centering
    \includegraphics[scale=0.5]{lab2/Lab02-Hinh6.png}
    \caption{Kết quả \texttt{ping} từ PC3 \texttt{(192.168.2.3)} đến PC1 \texttt{(192.168.1.2)}}
\end{figure}
\subsubsection{Cho biết bảng routing của Router R0 (có thể dùng GUI hoặc command)}
Để xem bảng routing của một Router, từ cửa sổ \texttt{CLI} ta nhập các lệnh sau:
\begin{lstlisting}
Router0>enable
Router0#sh ip route
\end{lstlisting}
Kết quả thu được như hình dưới đây:
\begin{figure}[H]
    \centering
    \includegraphics[scale=0.5]{lab2/Lab02-Hinh7.png}
    \caption{Bảng routing của router R0}
\end{figure}
\subsubsection{Đường mạng nào được cài đặt static routing? Phân loại mức tin cậy và độ dài của đường mạng static route?}
\begin{enumerate}
\item Đường mạng nào được cài đặt static routing?

Để biết được đường mạng nào cài đặt static routing, nhập lệnh sau vào \texttt{CLI} của cả 2 router:
\begin{lstlisting}
Router>enable
Router#show ip route static
\end{lstlisting}
Kết quả thu được như hình sau:
\begin{figure}[H]
    \centering
    \includegraphics[scale=0.5]{lab2/Lab02-Hinh8.png}
    \caption{Danh sách các đường mạng static routing của router R0}
\end{figure}
\begin{figure}[H]
    \centering
    \includegraphics[scale=0.5]{lab2/Lab02-Hinh9.png}
    \caption{Danh sách các đường mạng static routing của router R1}
\end{figure}
Từ đó, ta có thể rút ra kết luận: đường mạng được cài đặt static routing là đường nối 2 router R0 và R1, được tô đỏ dưới đây:
\begin{figure}[H]
    \centering
    \includegraphics[scale=0.5]{lab2/Lab02-Hinh10.png}
    \caption{Đường mạng static routing được tô đỏ.}
\end{figure}
\item Mức tin cậy\footnote{Administrative Distance} và độ dài \footnote{Metric} của đường mạng static route?
\begin{itemize}
\item Vì đường mạng này được cấu hình static routing nên mức tin cậy là \texttt{1}\cite{cisco:ad}.
\item Độ dài đường mạng chỉ được dùng trong các giao thức routing động\footnote{dynamic routing protocols}. Một đường mạng được cấu hình static route sẽ không có metric.\cite{cisco:metric}
\end{itemize}
\end{enumerate}
\section{Lab 3}
\subsection{Cách cài đặt}
Để cài đặt Lab3, nhóm đã thực hiện theo các bước sau:
\begin{enumerate}
\item Chọn Router, Switch và PC theo mô hình, sau đó nối dây theo đề bài.\footnote{Trong sơ đồ mạng có 2 vị trí \texttt{D1} và \texttt{D2} bị khuyết, giải thích trong ý \ref{lab03:loaitb}}

Chú ý là Router 2811 chỉ có 2 cổng \texttt{FastEthernet} mà theo sơ đồ thì các router cần ít nhất 3 cổng \texttt{FastEthernet} để thực hiện kết nối. Do đó, ta phải thực hiện lắp thêm cổng mạng cho router, ta làm theo các bước sau:
\begin{enumerate}
\item Chọn \texttt{Router > Physical}. Nhấn nút nguồn để tắt router.
\begin{figure}[H]
\centering
\includegraphics[scale=0.8]{lab3/lab03-hinh1.png}
\caption{Nút nguồn router}
\end{figure}
\item Trên thanh \texttt{Modules} bên trái, kéo thả module phù hợp vào vị trí bên trái. Ở đây nhóm chọn module \texttt{NM-ESW-161}\label{ganthem}
\begin{figure}[H]
\centering
\includegraphics[scale=0.8]{lab3/lab03-hinh2.png}
\caption{Kéo thả module vào vị trí thích hợp}
\end{figure}
\item Bật router lên trở lại.
\end{enumerate}
Cuối cùng, ta được sơ đồ mạng như hình sau:
\begin{figure}[H]
\centering
\includegraphics[scale=0.5]{lab3/lab03-hinh3.png}
\caption{Sơ đồ mạng Lab03}
\end{figure}
\item Cấu hình Server HTTP và Server DNS-DHCP.
\begin{enumerate}
\item Cấu hình Server HTTP
\begin{enumerate}
\item Ta tiến hành cấu hình IP tĩnh cho Server HTTP như ở mục \ref{cauhinhippc} của Lab 2.
\item Mở dịch vụ HTTP của Server HTTP lên: \texttt{Server > Services > HTTP}
\begin{figure}[H]
\centering
\includegraphics[scale=0.5]{lab3/lab03-hinh4.png}
\caption{Mở dịch vụ HTTP trên Server HTTP}
\end{figure}
\item Đặt DNS Server cho Server HTTP là IP của Server DNS-DHCP.
\begin{figure}[H]
\centering
\includegraphics[scale=0.5]{lab3/lab03-hinh5.png}
\caption{Đặt DNS Server cho Server HTTP}
\end{figure}
Đến đây, ta đã cấu hình xong HTTP Server.
\end{enumerate}
\item Cấu hình Server DNS-DHCP\label{lab3:cauhinhdnshttp}
\begin{enumerate}
\item Ta tiến hành cấu hình IP tĩnh cho Server DNS-DHCP như ở mục \ref{cauhinhippc} của Lab 2.
\item Đặt \texttt{Default Gateway} cho Server DNS-DHCP là địa chỉ đường mạng.
\begin{figure}[H]
\centering
\includegraphics[scale=0.5]{lab3/lab03-hinh6.png}
\caption{Cấu hình Default Gateway cho DNS-DHCP Server}
\end{figure}
\item Cấu hình dịch vụ DHCP: \texttt{Server > Services > DHCP}. Phần \texttt{Service} chọn \texttt{On}, sau đó tạo các cấu hình như hình và lưu lại.
\begin{figure}[H]
\centering
\includegraphics[scale=0.5]{lab3/lab03-hinh7.png}
\caption{Cấu hình server DHCP}
\end{figure}
\item Cấu hình dịch vụ DNS: \texttt{Server > Services > DNS}. Phần \texttt{DNS Services} chọn \texttt{On}. Tạo 2 record, một \texttt{CNAME} và một \texttt{A} như hình, sau đó lưu lại.
\begin{figure}[H]
\centering
\includegraphics[scale=0.5]{lab3/lab03-hinh8.png}
\caption{Cấu hình server DNS}
\end{figure}
\end{enumerate}
Đến đây, ta đã cấu hình DNS-DHCP server thành công.
\end{enumerate}
\item Cấu hình Router 2 chuyển hướng gói tin đến Router 1.

Ta cấu hình tương tự bước \ref{chRouter} trong Lab 2
\item Cấu hình Router 3 chuyển hướng gói tin đến Router 1.

Ta cấu hình tương tự bước \ref{chRouter} trong Lab 2
\item Cấu hình cho 2 Router 2 và 3 giao tiếp với nhau.

Vì 2 router đang nối với nhau qua cổng được gắn thêm ở mục \ref{ganthem}, ta phải cấu hình qua interface \texttt{vlan1} \textit{vlan mặc định} của cổng mà 2 router này nối với nhau.\cite{cisco:vlan} Cách làm này giống với bước \ref{chRouter}, nhưng ta thay đoạn code \texttt{CLI} bằng
\begin{lstlisting}
Router>enable
Router#conf t
Router(config) interface vlan1
... (tiep tuc config ip nhu buoc tren)
\end{lstlisting}
\end{enumerate}
\begin{table}[H]
    \centering
    \caption{Bảng routing của các router ở Lab 3}\label{lab3:static}
    \begin{tabular}[t]{lccc}
        \toprule
        Router&Network&Mask&Next Hop\\
        \midrule
        R1&172.16.1.0&255.255.255.0&192.168.1.2\\
          &172.16.2.0&255.255.255.0&192.168.2.2\\
        \midrule
        R2&172.16.3.0&255.255.255.0&192.168.1.1\\
          &172.16.2.0&255.255.255.0&192.168.3.2\\
        \midrule
        D2&172.16.3.0&255.255.255.0&192.168.2.1\\
          &172.16.1.0&255.255.255.0&192.168.3.1\\
        \bottomrule
    \end{tabular}
\end{table}
\subsection{Kiểm thử}
\subsubsection{Kiểm thử các PC đã được cấp phát DHCP IP hay chưa}
Thực hiện tương tự bước \ref{capPhatDHCP}, ta thu được kết quả ở \ref{lab03:ketQuaDHCP}
\subsubsection{Kiểm thử các PC có thể truy cập trang web hay chưa}
Thực hiện truy cập vào website \href{https://www.network.com}{https://www.network.com}. Kết quả xem tại \ref{lab3:truyCapWeb}
\subsubsection{Kiểm thử các mạng \texttt{172.16.1.0} và \texttt{172.16.2.0} có thể giao tiếp với nhau hay chưa}
Thực hiện \texttt{ping} từ bất kỳ máy tính nào trong mạng \texttt{172.16.1.0} đến mạng \texttt{172.16.2.0} và ngược lại. Kết quả xem ở \ref{lab3:resultPing}
\subsection{Báo cáo}
\subsubsection{Xác định loại thiết bị của D1, D2 để hoàn thành sơ đồ mạng.}\label{lab03:loaitb}
Xét theo cấu trúc của mạng \texttt{172.16.2.0}, vị trí D1 phải là một \texttt{switch}. Tương tự, xét theo cấu trúc của mạng \texttt{172.16.1.0} thì vị trí D2 phải là một \texttt{router}.
\subsubsection{Cấu hình DNS Server và Webserver cho địa chỉ \texttt{www.network.com} và kiểm thử truy cập bằng trình duyệt.}
\begin{enumerate}
\item Thực hiện các bước ở mục \ref{lab3:cauhinhdnshttp} để cấu hình DNS Server và HTTP Server (webserver).
\item Truy cập vào trình duyệt của Server DNS:\label{browser}\footnote{Đến thao tác này ta chỉ mới cấu hình DNS Server và Webserver, chưa cấu hình DHCP Server hay định tuyến các Router. Vì vậy, đến thời điểm này chỉ có thể truy cập vào website thông qua một máy khác cùng mạng để kiểm thử.}
\begin{figure}[H]
\centering
\includegraphics[scale=0.6]{lab3/lab03-hinh9.png}
\caption{Mở trình duyệt}
\end{figure}
\item Truy cập vào \texttt{www.network.com}. Đây là kết quả, chứng tỏ webserver đã chạy thành công.
\begin{figure}[H]
\centering
\includegraphics[scale=0.6]{lab3/lab03-hinh10.png}
\caption{Giao diện trang www.network.com}
\end{figure}
\end{enumerate}
\subsubsection{Cấu hình DHCP Server để cấp phát IP cho PC0, PC1, PC2 và PC3.}\label{lab03:ketQuaDHCP}
\begin{enumerate}
\item Ta tiến hành cấu hình DHCP cho server theo các bước ở mục \ref{lab3:cauhinhdnshttp}
\item Sau khi tiến hành cấu hình, truy cập vào giao diện \texttt{IP Configuration} để yêu cầu cấp phát DHCP (xem lại mục \ref{checkDHCP}).
\end{enumerate}
Đây là kết quả cấp phát DHCP IP của 4 PC:
\begin{figure}[H]
\centering
\includegraphics[scale=0.5]{lab3/lab03-hinh11.png}
\caption{IP máy PC0}
\end{figure}
\begin{figure}[H]
\centering
\includegraphics[scale=0.55]{lab3/lab03-hinh12.png}
\caption{IP máy PC1}
\end{figure}
\begin{figure}[H]
\centering
\includegraphics[scale=0.55]{lab3/lab03-hinh13.png}
\caption{IP máy PC2}
\end{figure}
\begin{figure}[H]
\centering
\includegraphics[scale=0.55]{lab3/lab03-hinh14.png}
\caption{IP máy PC3}
\end{figure}
\subsubsection{Cấu hình static routing cho các router để các mạng có thể giao tiếp với nhau.}
Cấu hình static routing đã được nêu trong bảng \ref{lab3:static}.
\subsubsection{Thực hiện lệnh \texttt{ping} và chụp screenshot kết quả \texttt{ping} từ PC0 tới PC2, từ PC3 tới PC1.}\label{lab3:resultPing}
Thao tác \texttt{ping}, xem lại mục \ref{howToPing}. Sau đây là kết quả thao tác theo đề bài:
\begin{figure}[H]
\centering
\includegraphics[scale=0.55]{lab3/lab03-hinh15.png}
\caption{Kết quả ping từ PC0 đến PC2}
\end{figure}
\begin{figure}[H]
\centering
\includegraphics[scale=0.55]{lab3/lab03-hinh16.png}
\caption{Kết quả ping từ PC3 đến PC1}
\end{figure}
\subsubsection{Truy cập \texttt{www.network.com} từ trình duyệt của PC1 và PC3.}\label{lab3:truyCapWeb}
Cách mở trình duyệt tương tự mục \ref{browser}. Sau đây là kết quả thao tác theo đề bài:
\begin{figure}[H]
\centering
\includegraphics[scale=0.55]{lab3/lab03-hinh17.png}
\caption{Truy cập vào \texttt{www.network.com} từ PC1}
\end{figure}
\begin{figure}[H]
\centering
\includegraphics[scale=0.55]{lab3/lab03-hinh18.png}
\caption{Truy cập vào \texttt{www.network.com} từ PC3}
\end{figure}
\begin{thebibliography}{10}
\bibitem{cisco:ad}
Cisco, \textit{What Is Administrative Distance?}, \href{https://www.cisco.com/c/en/us/support/docs/ip/border-gateway-protocol-bgp/15986-admin-distance.html}{https://www.cisco.com/c/en/us/support/docs/ip/border-gateway-protocol-bgp/15986-admin-distance.html}, truy cập \today.
\bibitem{cisco:metric}
Richard Burts via Cisco Community, \textit{Default Metric value for static routes in CISCO ?}, \href{https://community.cisco.com/t5/switching/default-metric-value-for-static-routes-in-cisco/td-p/2544008}{https://community.cisco.com/t5/switching/default-metric-value-for-static-routes-in-cisco/td-p/2544008}, truy cập \today.
\bibitem{cisco:vlan}
Cisco, \textit{Chapter: Configuring VLAN Interfaces }, \href{https://www.cisco.com/c/en/us/td/docs/app\_ntwk\_services/data\_center\_app_services/ace\_appliances/vA1\_7\_/configuration/routing\_bridging/guide/rtbrgdgd/vlansif.html}{Chapter: Configuring VLAN Interfaces}, truy cập \today.
\end{thebibliography}
\end{document}