\documentclass[12pt]{article}

\usepackage{amsmath}
\usepackage{amsfonts}
\usepackage{float}
\usepackage{fancyhdr}
\usepackage{graphicx}
\usepackage[colorlinks=true,linkcolor=blue, citecolor=red]{hyperref}
\usepackage{url}
\usepackage[top=1in, left=.5in, right=.5in, bottom=.75in]{geometry}
\usepackage[utf8]{vietnam}

\graphicspath{{./do-an-1-result/}}

\pagestyle{fancy}
\lhead{
	CSC14003 - Cơ sở trí tuệ nhân tạo\\
	Báo cáo đồ án 1: Các thuật toán tìm kiếm
}
\rhead{Trường Đại học Khoa học Tự nhiên - ĐHQG HCM}
\lfoot{Trần Hoàng Quân - Nguyễn Đắc Thắng}

\title{
	CSC14003 - Cơ sở trí tuệ nhân tạo\\
	Báo cáo đồ án 1: Các thuật toán tìm kiếm
}
\author{Trần Hoàng Quân - Nguyễn Đắc Thắng}

\begin{document}
\maketitle
\tableofcontents
\pagebreak

\section{Thông tin}
\subsection{Thông tin nhóm}
\begin{table}[ht]
\centering
\begin{tabular}{|c|c|}
\hline
Họ \& tên sinh viên & MSSV \\
\hline
Trần Hoàng Quân & 19120338 \\
Nguyễn Đắc Thắng & 19120364 \\
\hline
\end{tabular}
\end{table}

\subsection{Thông tin chương trình}
Chi tiết cách cài đặt và kiểm thử, mời xem file \texttt{Source/readme.md}
\section{Tìm kiếm trên bản đồ không có điểm thưởng}
\subsection{Giới thiệu}
Đối với các thuật toán tìm kiếm có thông tin, nhóm định nghĩa hàm heuristic $h(x)$ là khoảng cách từ điểm $x$ đến vị trí dừng EXIT theo khoảng cách Manhattan. Cụ thể hơn, giả sử ta có $x = (a, b)$ là tọa độ ô hiện tại và $(u, v)$ là tọa độ vị trí EXIT. Khi đó $h(x) = |a-u| + |b-v|$.
\\\\
Các thuật toán được cài đặt ở các vị trí sau:
\begin{itemize}
\item DFS: \texttt{Source/implementation/DFS.py}
\item BFS: \texttt{Source/implementation/BFS.py}
\item Greedy-BFS: \texttt{Source/implementation/greedy.py}
\item A-Star: \texttt{Source/implementation/astar.py}
\end{itemize}
Quá trình cài đặt có tham khảo nguồn \cite{redblob}


\subsection{Kiểm thử và nhận xét}
\subsubsection{Bản đồ 01 (\texttt{sample/01.txt})}
\begin{figure}[H]
\centering
\includegraphics[scale=0.8]{01-map.png}
\caption{Bản đồ 01}
\end{figure}

\begin{figure}[H]
\centering
\includegraphics[scale=0.8]{01-dfs.png}
\caption{Đường đi sử dụng thuật toán DFS}
\end{figure}

\begin{figure}[H]
\centering
\includegraphics[scale=0.8]{01-bfs.png}
\caption{Đường đi sử dụng thuật toán BFS}
\end{figure}

\begin{figure}[H]
\centering
\includegraphics[scale=0.8]{01-greedy.png}
\caption{Đường đi sử dụng thuật toán Greedy BFS.}
\end{figure}

\begin{figure}[H]
\centering
\includegraphics[scale=0.8]{01-astar.png}
\caption{Đường đi sử dụng thuật toán A-Star}
\end{figure}


Nhận xét: 
\begin{itemize}
\item DFS là thuật toán có đường đi dài nhất vì theo nguyên lý DFS sẽ duyệt xa nhất có thể. Ta có thể dễ dàng nhận thấy DFS duyệt những ngách trống rồi sau đó lùi ra, thay vì tránh ngách trống đó và về đích.
\item BFS và A-Star có đường đi như nhau, tuy nhiên vì A-Star đi theo heuristic nên sẽ rẽ ở ngã 3 gần đích, trong khi BFS chọn đi thắng vì thứ tự của queue.
\item Greedy-BFS có đường đi tương đổi giống với A-Star và BFS. Tuy nhiên, Greedy-BFS "mắc bẫy" ở ngách gần đích khi tính toán heuristic không tối ưu và phải tốn chi phí rẽ ngược ra.
\end{itemize}

\subsubsection{Bản đồ 02 (\texttt{sample/02.txt})}
\begin{figure}[H]
	\centering
	\includegraphics[scale=0.7]{02-map.png}
	\caption{Bản đồ 02}
\end{figure}

\begin{figure}[H]
	\centering
	\includegraphics[scale=0.7]{02-dfs.png}
	\caption{Đường đi sử dụng thuật toán DFS}
\end{figure}

\begin{figure}[H]
	\centering
	\includegraphics[scale=0.7]{02-bfs.png}
	\caption{Đường đi sử dụng thuật toán BFS}
\end{figure}

\begin{figure}[H]
	\centering
	\includegraphics[scale=0.7]{02-greedy.png}
	\caption{Đường đi sử dụng thuật toán Greedy BFS}
\end{figure}

\begin{figure}[H]
	\centering
	\includegraphics[scale=0.7]{02-astar.png}
	\caption{Đường đi sử dụng thuật toán A-Star}
\end{figure}

Nhận xét:
\begin{itemize}
\item DFS vẫn cho ra đường đi dài nhất. Vì không gian trống rất nhiều nên DFS sẽ duyệt (gần như) tất cả các điểm này.
\item BFS và A-Star vẫn cho ra đường đi tương đối giống nhau, khác nhau ở ngả rẽ gần đích. Nguyên nhân vẫn như trên - heuristic và thứ tự push vào queue.
\item Greedy-BFS có xu hướng hướng về phía điểm Exit, chỉ khi gặp chướng ngại vật thì mới rẽ và tiếp tục.
\end{itemize}

\subsubsection{Bản đồ 03 (\texttt{sample/03.txt})}
\begin{figure}[H]
	\centering
	\includegraphics[scale=0.7]{03-map.png}
	\caption{Bản đồ 03}
\end{figure}

\begin{figure}[H]
	\centering
	\includegraphics[scale=0.7]{03-dfs.png}
	\caption{Đường đi sử dụng thuật toán DFS}
\end{figure}

\begin{figure}[H]
	\centering
	\includegraphics[scale=0.7]{03-bfs.png}
	\caption{Đường đi sử dụng thuật toán BFS}
\end{figure}

\begin{figure}[H]
	\centering
	\includegraphics[scale=0.8]{03-greedy.png}
	\caption{Đường đi sử dụng thuật toán Greedy BFS}
\end{figure}

\begin{figure}[H]
	\centering
	\includegraphics[scale=0.8]{03-astar.png}
	\caption{Đường đi sử dụng thuật toán A-Star}
\end{figure}

Nhận xét:
\begin{itemize}
\item DFS vẫn có đường đi phức tạp nhất. Không gian trống càng nhiều thì đường đi kết quả của DFS càng dài.
\item BFS cho ra đường đi khác hướng với các thuật còn lại. Nguyên nhân chủ yếu vẫn là thứ tự push vào queue.
\item Greedy-BFS thể hiện rõ sự kém tối ưu: khi không gặp chướng ngại vật thì Greedy-BFS di chuyển theo heuristic hướng đến Exit, nhưng trong trường hợp này có chướng ngại vật bao quanh không gian trống. Khi đó ta "mắc kẹt" và phải tìm đường trở ngược ra, nên tốn kém chi phí.
\item A-Star tối ưu chi phí di chuyển nên không bị "mắc kẹt" như Greedy-BFS mặc dù cả hai thuật toán dùng chung một hàm heuristic.
\end{itemize}

\subsubsection{Bản đồ 04 (\texttt{sample/04.txt})}
\begin{figure}[H]
	\centering
	\includegraphics[scale=0.8]{04-map.png}
	\caption{Bản đồ 04}
\end{figure}

\begin{figure}[H]
	\centering
	\includegraphics[scale=0.8]{04-dfs.png}
	\caption{Đường đi sử dụng thuật toán DFS}
\end{figure}

\begin{figure}[H]
	\centering
	\includegraphics[scale=0.8]{04-bfs.png}
	\caption{Đường đi sử dụng thuật toán BFS}
\end{figure}

\begin{figure}[H]
	\centering
	\includegraphics[scale=0.8]{04-greedy.png}
	\caption{Đường đi sử dụng thuật toán Greedy BFS}
\end{figure}

\begin{figure}[H]
	\centering
	\includegraphics[scale=0.8]{04-astar.png}
	\caption{Đường đi sử dụng thuật toán A-Star}
\end{figure}

Nhận xét:
\begin{itemize}
\item DFS vẫn là thuật toán có đường đi dài nhất, phức tạp nhất.
\item BFS và Greedy-BFS có đường đi gần giống nhau. Khác biệt ở chỗ thứ tự duyệt của BFS quyết định nên đi sang phải trước; trong khi hàm heuristic làm cho Greedy-BFS quyết định đi sang trái trước. 
\item A-Star dựa trên heuristic và tối ưu đường đi nên khác với các kết quả còn lại.
\end{itemize}

\subsubsection{Bản đồ 05 (\texttt{sample/05.txt})}
\begin{figure}[H]
	\centering
	\includegraphics[scale=0.8]{05-map.png}
	\caption{Bản đồ 05}
\end{figure}

\begin{figure}[H]
	\centering
	\includegraphics[scale=0.8]{05-dfs.png}
	\caption{Đường đi sử dụng thuật toán DFS}
\end{figure}

\begin{figure}[H]
	\centering
	\includegraphics[scale=0.8]{05-bfs.png}
	\caption{Đường đi sử dụng thuật toán BFS}
\end{figure}

\begin{figure}[H]
	\centering
	\includegraphics[scale=0.8]{05-greedy.png}
	\caption{Đường đi sử dụng thuật toán Greedy BFS}
\end{figure}

\begin{figure}[H]
	\centering
	\includegraphics[scale=0.8]{05-astar.png}
	\caption{Đường đi sử dụng thuật toán A-Star}
\end{figure}

Nhận xét:
\begin{itemize}
\item DFS đã chọn đường đi đến đích dài nhất, tuy nhiên lại bỏ qua đường đi ngắn nhất.
\item Greedy-BFS tuân theo heuristic nhưng đường đi chưa thật sự tối ưu. Đây là một vấn đề thường xuất hiện ở những bản đồ có cổng ra gần với điểm xuất phát nhưng bị ngăn cách bởi chướng ngại vật. Greedy-BFS có xu hướng tuân theo heuristic một cách không tối ưu, dẫn đến việc "hấp tấp" tìm đường ra rồi tốn chi phí để tìm đường vòng "né" chướng ngại vật.
\item BFS và A-Star có đường đi gần giống, vẫn sai khác ở một số ngã rẽ vì thứ tự push vào queue và heuristic.
\end{itemize}

\section{Tìm kiếm trên bản đồ có điểm thưởng}
\subsection{Giới thiệu}
Với trường hợp bản đồ có điểm thưởng, nhóm quy về bài toán tìm đường đi ngắn nhất trên đồ thị có cung âm. Ta định nghĩa $\mathbb{B}$ là tập các điểm thưởng; $w_x$ là điểm thưởng nếu đi vào ô $x\ (x \in \mathbb{B})$. Hàm heuristic $h(a, b)$ được định nghĩa như sau:
$$
h(a, b) = \begin{cases}
w_b, & \text{if $b \in \mathbb{B}$}\\
|min_w(\mathbb{B})|, & \text{otherwise}
\end{cases}
$$
Giải thích: Ta phải thêm một lượng $|min_w(\mathbb{B})|$ vào để nâng trọng số các cạnh trong đồ thị, đưa về đồ thị có trọng số không âm. Nếu $x$ là một điểm thưởng, ta cộng trọng số vào chi phí đường đi để làm giảm chi phí. Như vậy, tác nhân di chuyển sẽ có xu hướng đi về các điểm thưởng.
\\\\
Tuy nhiên, ta phải tránh di chuyển về các hướng đã di chuyển. Nếu đi về các hướng đã di chuyển thì khả năng cao ta sẽ tạo thành một chu trình kín, dẫn đến lặp vô hạn. Ta ngăn chặn điều này bằng mảng đánh dấu, đánh dấu những điểm đã ghé thăm.
\\\\
Về bản chất, đây là thuật toán BFS kết hợp tối ưu hóa chi phí di chuyển bằng heuristic. Source code thuật toán nằm ở \texttt{Source/implementation/bonus.py}
\subsection{Kiểm thử}
\subsubsection{Bản đồ có 02 điểm thưởng (\texttt{sample/bonus-02.txt})}
\begin{figure}[H]
	\centering
	\includegraphics[scale=0.8]{bonus-02-map.png}
	\caption{Bản đồ có 02 điểm thưởng}
\end{figure}

\begin{figure}[H]
	\centering
	\includegraphics[scale=0.8]{bonus-02.png}
	\caption{Kết quả thuật toán trên bản đồ có 02 điểm thưởng}
\end{figure}

\begin{itemize}
	\item Tổng bonus ăn được: 1/2
	\item Tổng bonus: -9
	\item Chi phí naive: 37
	\item Tổng chi phí: 28
\end{itemize}

\subsubsection{Bản đồ có 05 điểm thưởng (\texttt{sample/bonus-05.txt})}
\begin{figure}[H]
	\centering
	\includegraphics[scale=0.8]{bonus-05-map.png}
	\caption{Bản đồ có 05 điểm thưởng}
\end{figure}

\begin{figure}[H]
	\centering
	\includegraphics[scale=0.8]{bonus-05.png}
	\caption{Kết quả thuật toán trên bản đồ có 05 điểm thưởng}
\end{figure}

\begin{itemize}
	\item Tổng bonus ăn được: 2/5
	\item Tổng bonus: -9
	\item Chi phí naive: 37
	\item Tổng chi phí: 28
\end{itemize}

\subsubsection{Bản đồ có 10 điểm thưởng (\texttt{sample/bonus-10.txt})}
\begin{figure}[H]
	\centering
	\includegraphics[scale=0.8]{bonus-10-map.png}
	\caption{Bản đồ có 10 điểm thưởng}
\end{figure}

\begin{figure}[H]
	\centering
	\includegraphics[scale=0.8]{bonus-10.png}
	\caption{Kết quả thuật toán trên bản đồ có 10 điểm thưởng}
\end{figure}

\begin{itemize}
\item Tổng bonus ăn được: 6/10
\item Tổng bonus: -44
\item Chi phí naive: 37
\item Tổng chi phí: 0 (Tổng bonus > chi phí naive)
\end{itemize}


\subsection{Nhận xét}
Dễ dàng nhận thấy thuật toán mà nhóm đề xuất ít có các bước đi lùi, mà chỉ tiến một hướng về đích. Bằng cách nâng các trọng số của các điểm bonus ta có thể hướng đường đi đi qua một số điểm bonus nhất định.
\\\\
Mặc dù vậy, thuật toán cũng đã thể hiện độ hiệu quả, cho ra chi phí di chuyển chấp nhận được. Nhóm sẽ tiếp tục cải tiến để mang lại hiệu quả cao hơn.

\bibliographystyle{plain}
\bibliography{do_an_1}

\end{document}
