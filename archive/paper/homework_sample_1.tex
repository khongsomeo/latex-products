\documentclass{article}

\usepackage{hyperref}
\usepackage{fancyhdr}
\usepackage{amsmath, amsfonts, amssymb}

\pagestyle{fancy}
\lhead{Homework Sample in \LaTeX}
\rhead{\LaTeX by Quan, Tran Hoang}

\begin{document}
    \section{Introduction}

    This document is a sample, from a real Linear Algebra practice homework given by the VNUHCM - University of Science.
    \\\\
    All works here belong to Quan, Tran Hoang. You can have a look at \href{https://github.com/trhgquan}{his GitHub page} or visit him at \href{https://www.tranhoangquan.codes}{his blog}.
    \\\\
    Notice that this file is a "example only", means it is not a real mathematics report. We just use this file to practice typing math formulas and page formatting.

    \section{Problem and Solution}
        \subsection{Problem 1.2(b)}
            \subsubsection{Statement}
                Multiply matrices:
                \begin{equation*}
                    \begin{bmatrix}1 & 1 & 1 \\ 0 & 1 & -2\end{bmatrix}
                    \begin{bmatrix}1 & -1 & 1 \\ -1 & 1 & -1 \\ 2 & 0 & -3\end{bmatrix}
                    \begin{bmatrix}3 & 1 \\ -1 & 2 \\ 4 & -3\end{bmatrix}
                \end{equation*}

            \subsubsection{Solution}
                First we multiplies first two matrices
                \begin{equation*}
                    \begin{bmatrix}1 & 1 & 1 \\ 0 & 1 & -2\end{bmatrix}
                    \begin{bmatrix}1 & -1 & 1 \\ -1 & 1 & -1 \\ 2 & 0 & -3\end{bmatrix}
                \end{equation*}
                \\
                which give the result
                \begin{equation*}
                    \begin{bmatrix}2 & 0 & -3 \\ -5 & 1 & 5\end{bmatrix}
                \end{equation*}
                \\
                Then multiplies with last matrix:
                \begin{equation*}
                    \begin{bmatrix}2 & 0 & -3 \\ -5 & 1 & 5\end{bmatrix}
                    \begin{bmatrix}3 & 1 \\ -1 & 2 \\ 4 & -3\end{bmatrix}
                \end{equation*}
                \\
                which gives the result
                \begin{equation*}
                    \begin{bmatrix}-6 & 11 \\ 4 & -18\end{bmatrix}
                \end{equation*}

        \subsection{Problem 1.3(c)}
            \subsubsection{Statement}
                Calculate $\mathbf{AA}^\intercal$ and $\mathbf{{A}^\intercal A}$, given
                \begin{equation*}
                    \mathbf{A} = \begin{bmatrix}3 & 2 & 2 \\ -1 & 3 & 4 \\ 9 & 1 & 2\end{bmatrix}
                \end{equation*}

            \subsubsection{Solution}
                First we calculate the $\mathbf{A}^\intercal$:
                \begin{equation*}
                    \mathbf{A}^\intercal = \begin{bmatrix}3 & -1 & 9 \\ 2 & 3 & 1 \\ 2 & 4 & 2\end{bmatrix}
                \end{equation*}
                \\
                Hence, we can easily calculate the product $\mathbf{AA}^\intercal$ and $\mathbf{{A}^\intercal A}$:
                \begin{equation*}
                    \mathbf{AA}^\intercal =
                    \begin{bmatrix}3 & 2 & 2 \\ -1 & 3 & 4 \\ 9 & 1 & 2\end{bmatrix}
                    \begin{bmatrix}3 & -1 & 9 \\ 2 & 3 & 1 \\ 2 & 4 & 2\end{bmatrix}
                    =
                    \begin{bmatrix}17 & 11 & 33 \\ 11 & 26 & 2 \\ 33 & 2 & 86\end{bmatrix}
                \end{equation*}
                \begin{equation*}
                    \mathbf{{A}^\intercal A} =
                    \begin{bmatrix}3 & -1 & 9 \\ 2 & 3 & 1 \\ 2 & 4 & 2\end{bmatrix}
                    \begin{bmatrix}3 & 2 & 2 \\ -1 & 3 & 4 \\ 9 & 1 & 2\end{bmatrix}
                    =
                    \begin{bmatrix}91 & 12 & 20 \\ 12 & 14 & 18 \\ 20 & 18 & 24\end{bmatrix}
                \end{equation*}

        \subsection{Problem 1.4(b)}
            \subsubsection{Statements}
                Calculate the result of the expression $\mathbf{AB - BA}$, given
                \begin{equation*}
                    \mathbf{A} = \begin{bmatrix}1 & 2 \\ 4 & -1\end{bmatrix}
                    ,
                    \mathbf{B} = \begin{bmatrix}2 & -3 \\ -4 & 1\end{bmatrix}
                \end{equation*}

            \subsubsection{Solution}
                First we need to calculate $\mathbf{AB}$ and $\mathbf{BA}$:
                \begin{equation*}
                    \mathbf{AB} =
                    \begin{bmatrix}1 & 2 \\ 4 & -1\end{bmatrix}
                    \begin{bmatrix}2 & -3 \\ -4 & 1\end{bmatrix}
                    =
                    \begin{bmatrix}-6 & -1 \\ 12 & -13\end{bmatrix}
                \end{equation*}
                \begin{equation*}
                    \mathbf{BA} =
                    \begin{bmatrix}2 & -3 \\ -4 & 1\end{bmatrix}
                    \begin{bmatrix}1 & 2 \\ 4 & -1\end{bmatrix}
                    =
                    \begin{bmatrix}-10 & 7 \\ 0 & -9\end{bmatrix}
                \end{equation*}
                \\
                Hence, result of the expression $\mathbf{AB - BA}$ is
                \begin{equation*}
                    \mathbf{AB - BA} =
                    \begin{bmatrix}-6 & -1 \\ 12 & -13\end{bmatrix}
                    -
                    \begin{bmatrix}-10 & 7 \\ 0 & -9\end{bmatrix}
                    =
                    \begin{bmatrix}4 & -8 \\ 12 & -4\end{bmatrix}
                \end{equation*}
\end{document}